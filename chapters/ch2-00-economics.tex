\part{Economia}
\label{ch:economics}
\chapter*{Economia}

\begin{chapquote}{Lewis Carroll, \textit{Alice no País das Maravilhas}}
\enquote{Uma grande roseira imperava na entrada do jardim: as rosas que nela cresciam eram brancas, mas havia três jardineiros que se ocupavam em pintá-las de vermelho. Alice achou que aquilo era uma coisa estranha e aproximou-se para ver melhor\ldots}
\end{chapquote}

O dinheiro não cresce em árvores. Acreditar que sim é tolice, e nossos pais garantem que saibamos disso, repetindo esse ditado quase como um mantra. Somos incentivados a usar o dinheiro com sabedoria, a não gastá-lo de qualquer jeito e a economizá-lo nos bons momentos para nos ajudar nos momentos difíceis. Afinal, o dinheiro não cresce em árvores.

O Bitcoin me ensinou mais sobre dinheiro do que eu jamais pensei que precisaria saber. Por meio dele, fui forçado a explorar a história do dinheiro, dos bancos, e pesquisar sobre várias escolas de pensamento econômico e muitas outras coisas. A busca para entender o Bitcoin me levou por uma infinidade de caminhos, alguns dos quais tento explorar neste capítulo.

Nas primeiras sete lições, algumas das questões filosóficas abordadas pelo Bitcoin foram discutidas. As próximas sete lições darão uma olhada mais de perto no dinheiro e na economia.

~

\begin{samepage}
Parte~\ref{ch:economics} -- Economia:

\begin{enumerate}
  \setcounter{enumi}{7}
  \item Ignorância financeira
  \item Inflação
  \item Valor
  \item Dinheiro
  \item A história e a queda do dinheiro
  \item A insanidade das reservas fracionadas
  \item O dinheiro forte
\end{enumerate}
\end{samepage}

Mais uma vez, só serei capaz de passar por cima destes pontos. O Bitcoin não é apenas ambicioso, mas também é amplo e profundo em seu escopo, tornando impossível cobrir todos os tópicos relevantes em uma única lição, ensaio, artigo ou livro. Duvido que seja mesmo possível.

O Bitcoin é uma nova forma de dinheiro, o que torna o aprendizado de economia fundamental para entendê-lo. Lidando com a natureza da ação humana e as interações dos agentes econômicos, a economia é provavelmente uma das maiores e mais confusas peças do quebra-cabeça do Bitcoin.

Novamente, essas lições são uma exploração das diversas coisas que aprendi com o Bitcoin. Elas são um reflexo pessoal de minha jornada pela toca do coelho. Não tendo formação em economia, estou definitivamente fora da minha zona de conforto e especialmente ciente de que qualquer compreensão que possa ter está incompleta. Farei o meu melhor para delinear o que aprendi, mesmo correndo o risco de me fazer de idiota. Afinal, ainda estou tentando responder à pergunta: \textit{\enquote{O que você aprendeu com o Bitcoin?}}

Depois de sete lições examinadas pelas lentes da filosofia, vamos usar as lentes da economia para examinar mais sete. A aula de economia é tudo que posso oferecer neste momento. Destino final: Uma \textit{moeda forte}.

% [the question]: https://twitter.com/arjunblj/status/1050073234719293440
