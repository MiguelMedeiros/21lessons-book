% !TEX root = main.tex
% \documentclass{tufte-book}%[a4paper,twoside]
% See https://github.com/Tufte-LaTeX/tufte-latex/blob/master/sample-book.tex for details

% --- AMAZON BEGIN ---
% WITHOUT BLEED
% US Trade => 6x9
\documentclass[paper=6in:9in,pagesize=pdftex,
               headinclude=on,footinclude=on,12pt]{scrbook}
%
% Paper width
% W = 6in
% Paper height
% H = 9in
% Paper gutter
% BCOR = 0.5in
% Margin (0.5in imposed on lulu, recommended on createspace)
% m = 0.5in
% Text height
% h = H - 2m = 8in
% Text width
% w = W - 2m - BCOR = 4.5in
\areaset[0.50in]{4.5in}{8in}
% --- AMAZON END ---

%Portuguese-specific commands
%--------------------------------------
\usepackage[portuguese]{babel}
%--------------------------------------

% Copyright with title BEGIN
\usepackage{fancyhdr}
\def\secondpage{\clearpage\null\vfill
\pagestyle{empty}
\begin{minipage}[b]{0.9\textwidth}
\normalsize As 21 Lições \newline
\footnotesize O Que Aprendi ao Cair na Toca do Coelho do Bitcoin \par

Primeira Edição. Versão 0.3.11, git commit \texttt{6a933bb}.

\footnotesize\raggedright
\setlength{\parskip}{0.5\baselineskip}
Copyright \copyright 2018--\the\year\ Gigi / \href{https://twitter.com/dergigi}{@dergigi} / \href{https://dergigi.com}{dergigi.com} \par

\footnotesize\raggedright
\setlength{\parskip}{0.5\baselineskip}
Tradução feita por KoreaComK / \href{https://twitter.com/koreacomk}{@KoreacomK} \par com ajuda do The Cypherpunk Pirate / \href{https://twitter.com/fllamber}{@fllamber}  \par

\includegraphics[width=2cm]{assets/images/cc-by-sa.pdf}

Este livro e sua versão online são distribuídos nos termos da
Licença Creative Commons Attribution-ShareAlike 4.0. Uma cópia da referência desta
a licença pode ser encontrada no site oficial creative commons.\footnote{\url{https://creativecommons.org/licenses/by-sa/4.0}}

\end{minipage}
\vspace*{2\baselineskip}
\cleardoublepage
\rfoot{\thepage}}

\makeatletter
\g@addto@macro{\maketitle}{\secondpage}
\makeatother
% Copyright with title END

% Use serif font for chapters and parts
\setkomafont{disposition}{\bfseries}
\KOMAoptions{headings=small}

% Packages
\usepackage{setspace}
\usepackage{booktabs}
\usepackage{graphicx}
\setkeys{Gin}{width=\linewidth,totalheight=\textheight,keepaspectratio}
\graphicspath{{graphics/}}

%%
% For Quotes
\usepackage{csquotes}
\renewcommand\mkbegdispquote[2]{\makebox[0pt][r]{\textquotedblleft\,}}
\renewcommand\mkenddispquote[2]{\,\textquotedblright#2}

%%
% Just some sample text
\usepackage{lipsum}

%%
% For nicely typeset tabular material
\usepackage{booktabs}

%%
% Bibliography stuff: Biber, BibTex, BibLatex
%\usepackage[autostyle]{csquotes}
% \usepackage[
    % backend=biber,
    % style=authoryear-icomp,
    % sortlocale=de_DE,
    % natbib=true,
    % url=false,
    % doi=true,
    % eprint=false
% ]{biblatex}
% \usepackage[backend=biber]{biblatex}
\usepackage{url}
\usepackage{natbib}
\bibliographystyle{plain}

%%
% Hyperlinks
\usepackage[hidelinks]{hyperref}

%%
% For graphics / images
\usepackage{caption}
\usepackage{graphicx}
\setkeys{Gin}{width=\linewidth,totalheight=\textheight,keepaspectratio}
\graphicspath{{graphics/}}

% The fancyvrb package lets us customize the formatting of verbatim
% environments.  We use a slightly smaller font.
\usepackage{fancyvrb}
\fvset{fontsize=\normalsize}

%%
% Prints argument within hanging parentheses (i.e., parentheses that take
% up no horizontal space).  Useful in tabular environments.
\newcommand{\hangp}[1]{\makebox[0pt][r]{(}#1\makebox[0pt][l]{)}}

%%
% Prints an asterisk that takes up no horizontal space.
% Useful in tabular environments.
\newcommand{\hangstar}{\makebox[0pt][l]{*}}

%%
% Prints a trailing space in a smart way.
\usepackage{xspace}

% Prints the month name (e.g., January) and the year (e.g., 2008)
\newcommand{\monthyear}{%
  \ifcase\month\or January\or February\or March\or April\or May\or June\or
  July\or August\or September\or October\or November\or
  December\fi\space\number\year
}


% Prints an epigraph and speaker in sans serif, all-caps type.
\newcommand{\openepigraph}[2]{%
  %\sffamily\fontsize{14}{16}\selectfont
  \begin{fullwidth}
  \sffamily\large
  \begin{doublespace}
  \noindent\allcaps{#1}\\% epigraph
  \noindent\allcaps{#2}% author
  \end{doublespace}
  \end{fullwidth}
}

% Inserts a blank page
\newcommand{\blankpage}{\newpage\hbox{}\thispagestyle{empty}\newpage}

\usepackage{units}

% Typesets the font size, leading, and measure in the form of 10/12x26 pc.
\newcommand{\measure}[3]{#1/#2$\times$\unit[#3]{pc}}

% Macros for typesetting the documentation
\newcommand{\hlred}[1]{\textcolor{Maroon}{#1}}% prints in red
\newcommand{\hangleft}[1]{\makebox[0pt][r]{#1}}
\newcommand{\hairsp}{\hspace{1pt}}% hair space
\newcommand{\hquad}{\hskip0.5em\relax}% half quad space
\newcommand{\TODO}{\textcolor{red}{\bf TODO!}\xspace}
\newcommand{\na}{\quad--}% used in tables for N/A cells
\providecommand{\XeLaTeX}{X\lower.5ex\hbox{\kern-0.15em\reflectbox{E}}\kern-0.1em\LaTeX}
\newcommand{\tXeLaTeX}{\XeLaTeX\index{XeLaTeX@\protect\XeLaTeX}}
% \index{\texttt{\textbackslash xyz}@\hangleft{\texttt{\textbackslash}}\texttt{xyz}}
\newcommand{\tuftebs}{\symbol{'134}}% a backslash in tt type in OT1/T1
\newcommand{\doccmdnoindex}[2][]{\texttt{\tuftebs#2}}% command name -- adds backslash automatically (and doesn't add cmd to the index)
\newcommand{\doccmddef}[2][]{%
  \hlred{\texttt{\tuftebs#2}}\label{cmd:#2}%
  \ifthenelse{\isempty{#1}}%
    {% add the command to the index
      \index{#2 command@\protect\hangleft{\texttt{\tuftebs}}\texttt{#2}}% command name
    }%
    {% add the command and package to the index
      \index{#2 command@\protect\hangleft{\texttt{\tuftebs}}\texttt{#2} (\texttt{#1} package)}% command name
      \index{#1 package@\texttt{#1} package}\index{packages!#1@\texttt{#1}}% package name
    }%
}% command name -- adds backslash automatically
\newcommand{\doccmd}[2][]{%
  \texttt{\tuftebs#2}%
  \ifthenelse{\isempty{#1}}%
    {% add the command to the index
      \index{#2 command@\protect\hangleft{\texttt{\tuftebs}}\texttt{#2}}% command name
    }%
    {% add the command and package to the index
      \index{#2 command@\protect\hangleft{\texttt{\tuftebs}}\texttt{#2} (\texttt{#1} package)}% command name
      \index{#1 package@\texttt{#1} package}\index{packages!#1@\texttt{#1}}% package name
    }%
}% command name -- adds backslash automatically
\newcommand{\docopt}[1]{\ensuremath{\langle}\textrm{\textit{#1}}\ensuremath{\rangle}}% optional command argument
\newcommand{\docarg}[1]{\textrm{\textit{#1}}}% (required) command argument
\newenvironment{docspec}{\begin{quotation}\begin{samepage}\ttfamily\parskip0pt\parindent0pt\ignorespaces}{\end{flushright}\end{samepage}\end{quotation}}% command specification environment
\newcommand{\docenv}[1]{\texttt{#1}\index{#1 environment@\texttt{#1} environment}\index{environments!#1@\texttt{#1}}}% environment name
\newcommand{\docenvdef}[1]{\hlred{\texttt{#1}}\label{env:#1}\index{#1 environment@\texttt{#1} environment}\index{environments!#1@\texttt{#1}}}% environment name
\newcommand{\docpkg}[1]{\texttt{#1}\index{#1 package@\texttt{#1} package}\index{packages!#1@\texttt{#1}}}% package name
\newcommand{\doccls}[1]{\texttt{#1}}% document class name
\newcommand{\docclsopt}[1]{\texttt{#1}\index{#1 class option@\texttt{#1} class option}\index{class options!#1@\texttt{#1}}}% document class option name
\newcommand{\docclsoptdef}[1]{\hlred{\texttt{#1}}\label{clsopt:#1}\index{#1 class option@\texttt{#1} class option}\index{class options!#1@\texttt{#1}}}% document class option name defined
\newcommand{\docmsg}[2]{\bigskip\begin{fullwidth}\noindent\ttfamily#1\end{fullwidth}\medskip\par\noindent#2}
\newcommand{\docfilehook}[2]{\texttt{#1}\index{file hooks!#2}\index{#1@\texttt{#1}}}
\newcommand{\doccounter}[1]{\texttt{#1}\index{#1 counter@\texttt{#1} counter}}

% Generates the index
\usepackage{makeidx}
\makeindex

%%
% Chapter/Lesson Quotes
\makeatletter
\renewcommand{\@chapapp}{}% Not necessary...
\newenvironment{chapquote}[2][4em]
  {\setlength{\@tempdima}{#1}%
   \def\chapquote@author{#2}%
   \parshape 1 \@tempdima \dimexpr\textwidth-2\@tempdima\relax%
   \itshape}
  {\par\normalfont\hfill--\ \chapquote@author\hspace*{\@tempdima}\par\bigskip}
\makeatother

%%%%%%%%%%%%%%%%%%%%%%%%%%%%%%%%%%%%%%%%%%%%%%%%%%%%%%%%%%%%%%%%%%%%%%%%%%%%%%%%
%                                   DOCUMENT
%%%%%%%%%%%%%%%%%%%%%%%%%%%%%%%%%%%%%%%%%%%%%%%%%%%%%%%%%%%%%%%%%%%%%%%%%%%%%%%%

\begin{document}

\frontmatter

\title{As 21 Lições}
\subtitle{O Que Aprendi ao Cair na Toca do Coelho do Bitcoin}
\author{Gigi}
\date{}

\maketitle

\cleardoublepage

\input{frontmatter/dedication}
\chapter*{Introdução}
\pdfbookmark{Foreword}{foreword}

Alguns chamam isso de experiência religiosa. Outros o chamam de Bitcoin.

Encontrei o Gigi pela primeira vez em uma de minhas casas espirituais - em Riga, Letônia - a casa da Conferência chamada \textit{The Baltic Honeybadger}, onde os mais fervorosos fiéis do Bitcoin fazem uma peregrinação anual. Depois de uma conversa profunda na hora do almoço, o título que Gigi e eu forjamos estava tão fixo em nossas cabeças quanto uma transação de Bitcoin que foi processada quando apertamos as mãos pela primeira vez, algumas horas antes.

Meu outro lar espiritual, Christ Church, em Oxford, onde tive o privilégio de estudar para o meu MBA, foi onde tive meu momento do \enquote{Buraco do Coelho}. Como Gigi, transcendi os reinos econômicos, técnicos e sociais, e fui espiritualmente envolvido pelo Bitcoin. Depois de \enquote{comprar na alta} na bolha de novembro de 2013, havia várias lições extremamente difíceis a serem aprendidas no mercado que ficou em baixa por 3 anos, um tempo que parecia implacavelmente esmagador e interminável. Essas 21 lições realmente teriam me servido muito bem naquela época. Muitas dessas lições são simplesmente verdades naturais que, para aqueles que não foram iniciados, são obscurecidas por um papel opaco e frágil. Porém, no final deste livro, o papel irá se dissolver de maneira feroz.

Em uma noite cristalina em Oxford, no final de agosto de 2016, apenas algumas semanas depois da faca ter sido retorcida no meu coração quando o corretora Bitfinex foi hackeada, sentei-me em silenciosa contemplação no Jardim do Mestre da Igreja de Cristo. Os tempos eram difíceis e eu estava no meu limite mental e emocional depois do que parecia ser uma vida inteira de tortura. Não era devido a perda financeira, mas pela esmagadora perda espiritual que sentia por estar isolado em minha visão de mundo. Se ao menos houvesse os recursos que temos hoje naquela época eu saberia que não estava sozinho. O Jardim do Mestre é um lugar muito especial para mim e para muitos que vieram antes de mim ao longo dos séculos. Foi lá que Charles Dodgson, professor de matemática na Christ Church, observou uma de suas jovens alunas, Alice Liddell, filha do reitor da Christ Church. Dodgson, mais conhecido por seu pseudônimo, Lewis Carroll, usou Alice e o jardim como sua inspiração, e na magia daquela relva sagrada, eu encarei profundamente o cripto-abismo, e ele olhou de volta para mim, de maneira ardente, aniquilando minha arrogância, e dando um tapa na minha cara, acabando com o meu orgulho. Eu finalmente estava em paz.

As 21 lições leva você a uma verdadeira jornada ao Bitcoin, não apenas uma jornada ao mundo da filosofia, tecnologia e economia, mas da alma.

À medida que você se aprofunda na filosofia resumidamente apresentada nas 7 das 21 Lições, pode-se ir tão longe a ponto de compreender a origem de todos os seres com tempo e contemplação suficientes. Suas 7 lições de economia capturam, em termos simples, como estamos à mercê financeira de um pequeno grupo de \textit{Chapeleiros Malucos} e como eles conseguiram colocar cegar nossas mentes, corações e almas. As 7 lições sobre tecnologia mostram a beleza e a perfeição tecnologicamente darwiniana do Bitcoin. Sendo um Bitcoinheiro sem bagagem técnica, as lições fornecem uma revisão relevante da natureza tecnológica inerente do Bitcoin e, na verdade, da própria natureza da tecnologia.

Nessa experiência transitória que chamamos de vida, vivemos, amamos e aprendemos. Mas o que é a vida senão uma ordem de eventos com registro de data e hora?

Conquistar a montanha do Bitcoin não é fácil. Os cumes falsos são abundantes, as rochas são ásperas e as rachaduras e fendas estão à espreita para nos engolir. Depois de ler este livro, verá que Gigi é o Pastor do Bitcoin e eu o apreciarei para sempre.

\begin{flushright}
  Hass McCook \\
  November 29, 2019
\end{flushright}


\newpage \vspace*{4cm}
\thispagestyle{empty}
\begin{quotation}
\begin{center}
  \large
  \enquote{Poderia me dizer, por favor, que caminho devo tomar para sair daqui?} \\~\\
  \enquote{Isso depende bastante de onde você quer chegar.} \\~\\
  \enquote{O lugar não importa muito --} \\~\\
  \enquote{Então não importa o caminho que você vai tomar.}
\end{center}
\begin{flushright} -- Lewis Carroll, \textit{Alice no País das Maravilhas}\end{flushright}
\end{quotation}

\tableofcontents

\input{frontmatter/about}
\chapter*{Prefácio}

Cair na toca do coelho do Bitcoin é uma experiência estranha. Como muitos outros, sinto que aprendi mais nos últimos anos estudando sobre o Bitcoin do que durante duas décadas de educação formal.

As lições a seguir são uma compilado de tudo o que aprendi. Publicado pela primeira vez como uma série de artigos intitulada {“O que aprendi com o Bitcoin”}, o que se segue pode ser visto como uma terceira edição da série original.

Como o Bitcoin, essas lições não são estáticas. Pretendo trabalhar nelas periodicamente, lançando versões atualizadas e material adicional no futuro.

Ao contrário do Bitcoin, as versões futuras deste projeto não precisam ser compatíveis com versões anteriores. Algumas lições podem ser aumentadas, outras podem ser refeitas ou mesmo, substituídas.

O Bitcoin é um professor que não se cansa, por isso não posso afirmar que essas lições sejam abrangentes ou completas. Elas são um reflexo de minha jornada pessoal pela toca do coelho. Há mais lições a serem aprendidas e cada pessoa aprenderá algo diferente ao entrar no mundo do Bitcoin.

Espero que você ache essas lições úteis e que o processo de aprendizado lendo-as não seja tão árduo e doloroso quanto aprendê-las por você mesmo.

% <!-- Internal -->
% [I]: 
%
% <!-- Twitter -->
% [dergigi]: https://twitter.com/dergigi
%
% <!-- Wikipedia -->
% [alice]: https://en.wikipedia.org/wiki/Alice%27s_Adventures_in_Wonderland
% [carroll]: https://en.wikipedia.org/wiki/Lewis_Carroll

%%
% Start the main matter (normal chapters)
\mainmatter

\part*{As 21 Lições}
\input{quote}
\input{chapters/ch0-02-introduction}
\input{chapters/ch1-00-philosophy}
\chapter{Imutabilidade e mudança}
\label{les:1}

\begin{chapquote}{Alice}
\enquote{O que será que mudou à noite? Deixe-me ver: eu era a mesma quando acordei de manhã? Tenho a impressão de ter me sentido um pouco diferente. Mas se eu não sou a mesma, a próxima questão é “Quem sou eu?” Ah! esta é a grande confusão!}
\end{chapquote}

O Bitcoin é por padrão difícil de se descrever. É uma \textit {coisa nova}, e qualquer tentativa de fazer uma comparação com conceitos anteriores - seja chamando-o de ouro digital ou de dinheiro da internet  - está fadada a ficar aquém do todo. Qualquer que seja sua analogia favorita, dois aspectos do Bitcoin são absolutamente essenciais: descentralização e imutabilidade.

\paragraph{}
Uma maneira de pensar sobre o Bitcoin é como um contrato social automatizado \footnote{Hasu, Unpacking Bitcoin's Social Contract~\cite {social-contract1}}. O software é apenas uma peça do quebra-cabeça, e esperar mudar o Bitcoin mudando o software é um exercício de futilidade. Seria preciso convencer o restante da rede a adotar as mudanças, o que é mais um esforço psicológico do que de engenharia de software.

\paragraph{}
O que se segue pode parecer absurdo à primeira vista, como tantas outras coisas neste espaço, mas acredito que seja profundamente verdadeiro, no entanto: você não mudará o Bitcoin, mas Bitcoin irá mudar você.

\begin{quotation}\begin{samepage}
\enquote{O Bitcoin irá nos mudar mais do que nós podemos mudá-lo.}
\begin{flushright} -- Marty Bent\footnote{Tales From the Crypt~\cite{tftc21}}
\end{flushright}\end{samepage}\end{quotation}

Levei muito tempo para perceber a profundidade disso. Como o Bitcoin é apenas software e tudo é de código aberto, você pode simplesmente mudar as coisas à vontade, certo? Errado. \textit {Muito} errado. Sem nenhuma surpresa, o criador do Bitcoin sabia disso muito bem.

\begin{quotation}\begin{samepage}
\enquote{A natureza do Bitcoin é tal que no momento que a versão 0.1 foi lançada, o design do núcleo foi definido para o resto da vida.}
\begin{flushright} -- Satoshi Nakamoto\footnote{Postagem no fórum do BitcoinTalk: `Resposta: Transações e Scripts \ldots'~\cite{satoshi-set-in-stone}}
\end{flushright}\end{samepage}\end{quotation}

Muitas pessoas tentaram mudar a natureza do Bitcoin. Até agora, todos falharam. Embora exista um mar infinito de forks e altcoins, a rede Bitcoin ainda faz seu trabalho, assim como fazia quando o primeiro nó estava online. As altcoins não importarão no longo prazo. Os forks acabarão morrendo definhados. O Bitcoin é o que importa. Enquanto nosso entendimento fundamental de matemática e/ou física não mudar, o honeybadger do Bitcoin continuará a não se importar.


\begin{quotation}\begin{samepage}
\enquote{O Bitcoin é o primeiro exemplo de uma nova aforma de vida. Ela vive e respira na internet. Ela vive porque ela pode pagar as pessoas para que ela continue viva. [\ldots] Não pode ser mudada. Não podemos discutir com ela. Não pode ser adulterada. Não pode ser corrompida. Não pode ser parada. [\ldots] Se uma guerra nuclear destruir metade do nosso planeta, ela continuará viva e incorruptível.}
\begin{flushright} -- Ralph Merkle\footnote{DAOs, Democracy and
Governance,~\cite{merkle-dao}}
\end{flushright}\end{samepage}\end{quotation}

O batimento cardíaco da rede Bitcoin durará mais do que todos os nossos.

~

Depois que entendi a citação acima, ela me mudou muito mais do que os blocos anteriores do blockchain do Bitcoin jamais farão. Mudou minha preferência temporal, meu entendimento de economia, minhas visões políticas e muito mais. Maldição, está até mudando a dieta das pessoas \footnote{Inside the World of the Bitcoin
Carnivores,~\cite{carnivores}}. Se tudo isso parece loucura para você, não se preocupe, você está em boa companhia. Tudo isso é uma loucura e, no entanto, está acontecendo.

~

\paragraph{O Bitcoin me ensinou que ele não irá mudar. Eu irei.}

% ---
%
% #### Through the Looking-Glass
%
% - [Bitcoin's Gravity: How idea-value feedback loops are pulling people in][gravity]
% - [Lesson 18: Move slowly and don't break things][lesson18]
%
% #### Down the Rabbit Hole
%
% - [Unpacking Bitcoin's Social Contract][automated social contract]: A framework for skeptics by Hasu
% - [DAOs, Democracy and Governance][Ralph Merkle] by Ralph C. Merkle
% - [Marty's Bent][bent]: A daily newsletter highlighting signal in Bitcoin by Marty Bent
% - [Technical Discussion on Bitcoin's Transactions and Scripts][Satoshi Nakamoto] by Satoshi Nakamoto, Gavin Andresen, and others
% - [Inside the World of the Bitcoin Carnivores][carnivores]: Why a small community of Bitcoin users is eating meat exclusively by Jordan Pearson
% - [Tales From the Crypt][tftc] hosted by Marty Bent
%
% <!-- Internal -->
% [gravity]: 
% [lesson18]: {{ 'bitcoin/lessons/ch3-18-move-slowly-and-dont-break-things' | absolute_url }}
%
% <!-- Further Reading -->
% [automated social contract]: https://medium.com/@hasufly/bitcoins-social-contract-1f8b05ee24a9
% [carnivores]: https://motherboard.vice.com/en_us/article/ne74nw/inside-the-world-of-the-bitcoin-carnivores
% [tftc]: https://tftc.io/tales-from-the-crypt/
% [bent]: https://tftc.io/martys-bent/
%
% <!-- Quotes -->
% [Ralph Merkle]: http://merkle.com/papers/DAOdemocracyDraft.pdf
% [Satoshi Nakamoto]: https://bitcointalk.org/index.php?topic=195.msg1611#msg1611
%
% <!-- Twitter People -->
% [Marty Bent]: https://twitter.com/martybent
%
% <!-- Wikipedia -->
% [alice]: https://en.wikipedia.org/wiki/Alice%27s_Adventures_in_Wonderland
% [carroll]: https://en.wikipedia.org/wiki/Lewis_Carroll


\chapter{A escassez da escassez}
\label{les:2}

\begin{chapquote}{Alice}
\enquote{É o suficiente\ldots eu espero não crescer mais\ldots}
\end{chapquote}

Em geral, o avanço da tecnologia parece tornar as coisas mais abundantes. Cada vez mais pessoas podem desfrutar do que antes eram bens luxuosos. Em breve, todos nós viveremos como reis. A maioria de nós já vive assim. Como Peter Diamandis escreveu em Abundance~\cite{abundance}: \enquote{A tecnologia é um mecanismo de liberação de recursos. Pode tornar o que antes era escasso em abundante.}

O Bitcoin, uma tecnologia avançada em si, quebra essa tendência e cria uma nova commodity que é realmente escassa. Alguns até argumentam que é uma das coisas mais raras do universo. A oferta não pode ser inflacionada, não importa quanto esforço seja despendido para se criar mais.

\begin{quotation}\begin{samepage}
\enquote{Apenas duas coisas são genuinamente escassas: O Tempo e o Bitcoin.}
\begin{flushright} -- Saifedean Ammous\footnote{Apresentação do livro The Bitcoin Standard~\cite{bitcoinstandard-pres}}
\end{flushright}\end{samepage}\end{quotation}

Paradoxalmente, ele o faz por meio de um mecanismo de cópia. As transações são transmitidas, os blocos são propagados, o livro razão distribuído é --- bem, você adivinhou --- distribuído. Todas essas são apenas palavras bonitas para dizer a mesma coisa: copiar. Caramba, o Bitcoin até mesmo se copia em quantos computadores puder, incentivando pessoas individuais a executar nodes completos e a minerar novos blocos.

Toda essa duplicação funciona maravilhosamente em conjunto em um esforço concentrado para produzir escassez.

\paragraph{Em tempos de abundância, o Bitcoin me ensinou o que é a verdadeira escassez.}

% ---
%
% #### Through the Looking-Glass
%
% - [Lesson 14: Sound money][lesson14]
%
% #### Down the Rabbit Hole
%
% - [The Bitcoin Standard: The Decentralized Alternative to Central Banking][bitcoin-standard]
% - [Abundance: The Future Is Better Than You Think][Abundance] by Peter Diamandis
% - [Presentation on The Bitcoin Standard][bitcoin-standard-presentation] by Saifedean Ammous
% - [Modeling Bitcoin's Value with Scarcity][planb-scarcity] by PlanB
% - 🎧 [Misir Mahmudov on the Scarcity of Time & Bitcoin][tftc60] TFTC #60 hosted by Marty Bent
% - 🎧 [PlanB – Modelling Bitcoin's digital scarcity through stock-to-flow techniques][slp67] SLP #67 hosted by Stephan Livera
%
% <!-- Through the Looking-Glass -->
% [lesson14]: {{ 'bitcoin/lessons/ch2-14-sound-money' | absolute_url }}
%
% <!-- Down the Rabbit Hole -->
% [Abundance]: https://www.diamandis.com/abundance
% [bitcoin-standard]: http://amzn.to/2L95bJW
% [bitcoin-standard-presentation]: https://www.bayernlb.de/internet/media/de/ir/downloads_1/bayernlb_research/sonderpublikationen_1/bitcoin_munich_may_28.pdf
% [planb-scarcity]: https://medium.com/@100trillionUSD/modeling-bitcoins-value-with-scarcity-91fa0fc03e25
% [tftc60]: https://anchor.fm/tales-from-the-crypt/episodes/Tales-from-the-Crypt-60-Misir-Mahmudov-e3aibh
% [slp67]: https://stephanlivera.com/episode/67
%
% <!-- Wikipedia -->
% [alice]: https://en.wikipedia.org/wiki/Alice%27s_Adventures_in_Wonderland
% [carroll]: https://en.wikipedia.org/wiki/Lewis_Carroll

\input{lessons/ch1-03-replication-and-locality}
\input{lessons/ch1-04-the-problem-of-identity}
\input{lessons/ch1-05-an-immaculate-conception}
\input{lessons/ch1-06-the-power-of-free-speech}
\input{lessons/ch1-07-the-limits-of-knowledge}
\part{Economia}
\label{ch:economics}
\chapter*{Economia}

\begin{chapquote}{Lewis Carroll, \textit{Alice no País das Maravilhas}}
\enquote{Uma grande roseira imperava na entrada do jardim: as rosas que nela cresciam eram brancas, mas havia três jardineiros que se ocupavam em pintá-las de vermelho. Alice achou que aquilo era uma coisa estranha e aproximou-se para ver melhor\ldots}
\end{chapquote}

O dinheiro não cresce em árvores. Acreditar que sim é tolice, e nossos pais garantem que saibamos disso, repetindo esse ditado quase como um mantra. Somos incentivados a usar o dinheiro com sabedoria, a não gastá-lo de qualquer jeito e a economizá-lo nos bons momentos para nos ajudar nos momentos difíceis. Afinal, o dinheiro não cresce em árvores.

O Bitcoin me ensinou mais sobre dinheiro do que eu jamais pensei que precisaria saber. Por meio dele, fui forçado a explorar a história do dinheiro, dos bancos, e pesquisar sobre várias escolas de pensamento econômico e muitas outras coisas. A busca para entender o Bitcoin me levou por uma infinidade de caminhos, alguns dos quais tento explorar neste capítulo.

Nas primeiras sete lições, algumas das questões filosóficas abordadas pelo Bitcoin foram discutidas. As próximas sete lições darão uma olhada mais de perto no dinheiro e na economia.

~

\begin{samepage}
Parte~\ref{ch:economics} -- Economia:

\begin{enumerate}
  \setcounter{enumi}{7}
  \item Ignorância financeira
  \item Inflação
  \item Valor
  \item Dinheiro
  \item A história e a queda do dinheiro
  \item A insanidade das reservas fracionadas
  \item O dinheiro forte
\end{enumerate}
\end{samepage}

Mais uma vez, só serei capaz de passar por cima destes pontos. O Bitcoin não é apenas ambicioso, mas também é amplo e profundo em seu escopo, tornando impossível cobrir todos os tópicos relevantes em uma única lição, ensaio, artigo ou livro. Duvido que seja mesmo possível.

O Bitcoin é uma nova forma de dinheiro, o que torna o aprendizado de economia fundamental para entendê-lo. Lidando com a natureza da ação humana e as interações dos agentes econômicos, a economia é provavelmente uma das maiores e mais confusas peças do quebra-cabeça do Bitcoin.

Novamente, essas lições são uma exploração das diversas coisas que aprendi com o Bitcoin. Elas são um reflexo pessoal de minha jornada pela toca do coelho. Não tendo formação em economia, estou definitivamente fora da minha zona de conforto e especialmente ciente de que qualquer compreensão que possa ter está incompleta. Farei o meu melhor para delinear o que aprendi, mesmo correndo o risco de me fazer de idiota. Afinal, ainda estou tentando responder à pergunta: \textit{\enquote{O que você aprendeu com o Bitcoin?}}

Depois de sete lições examinadas pelas lentes da filosofia, vamos usar as lentes da economia para examinar mais sete. A aula de economia é tudo que posso oferecer neste momento. Destino final: Uma \textit{moeda forte}.

% [the question]: https://twitter.com/arjunblj/status/1050073234719293440

\input{lessons/ch2-08-financial-ignorance}
\input{lessons/ch2-09-inflation}
\input{lessons/ch2-10-value}
\input{lessons/ch2-11-money}
\chapter{A história e a queda do dinheiro}
\label{les:12}

\begin{chapquote}{Lewis Carroll, \textit{Alice no País das Maravilhas}}
\enquote{tudo porque não se lembravam das regrinhas simples que seus amigos lhes haviam ensinado: que um atiçador em brasa acaba queimando sua mão se você insistir em segurá-lo por muito tempo; quando você corta o dedo muito fundo com uma faca, geralmente sai sangue;e ela nunca esquecera que, se você bebe muito de uma garrafa em que está escrito “veneno”, é quase certo que vai se sentir mal, mais cedo ou mais tarde.}
\end{chapquote}

Muitas pessoas pensam que o dinheiro é lastreado em ouro, que está trancado em grandes cofres enterrado bem fundo no subterrâneo do banco central do país, protegido por paredes grossas. Isso deixou de ser verdade há muitas décadas. Não tenho certeza do que pensei, uma vez que estava em apuros bem sérios, não tendo virtualmente nenhuma compreensão de ouro, papel-moeda ou por que ele precisaria ser lastreado por algo em primeiro lugar.

Uma parte de aprender sobre o Bitcoin é aprender sobre a moeda fiduciária: o que significa, como surgiu e por que pode não ser a melhor ideia que já tivemos. Então, o que exatamente é a moeda fiduciária? E como acabamos usando isso?

Se algo é imposto \textit{fiduciariamente}, significa simplesmente que é imposto por autorização ou proposição formal. Assim, a moeda fiduciária é dinheiro simplesmente porque \textit{alguém} disse que é dinheiro. Como todos os governos usam moeda fiduciária hoje, esse alguém é \textit{seu} governo. Infelizmente, você não está \textit{livre} para discordar desta proposta de valor. Você sentirá rapidamente que essa proposição é tudo, menos pacífica. Se você se recusar a usar esse papel-moeda para fazer negócios e pagar impostos, as únicas pessoas com quem você poderá discutir sobre economia serão seus colegas de cela.

O valor da moeda fiduciária não decorre de suas propriedades inerentes. O quão bom é uma moeda fiduciária, está relacionado a, única e exclusivamente, sua capacidade de ter (in)estabilidade política e fiscal daqueles que sonham com sua existência. Seu valor é imposto por decreto, de forma arbitrária.

\begin{figure}
  \centering
  \includegraphics[width=8cm]{assets/images/fiat-definition.png}
  \caption{Fiat --- `Deixe ser feito'}
  \label{fig:fiat-definition}
\end{figure}

\paragraph{}
Até recentemente, dois tipos de dinheiro eram usados: \textbf{moeda-mercadoria}, feita com \textit{coisas} preciosas, e \textbf{dinheiro representativo}, que simplesmente \textit{representa} o que é precioso, principalmente na escrita.

\paragraph{}
Já tocamos na moeda-mercadoria acima. As pessoas usavam ossos especiais, conchas e metais preciosos como dinheiro. Mais tarde, principalmente moedas feitas de metais preciosos como ouro e prata foram usadas como dinheiro. A moeda mais antiga encontrada até agora é feita de uma mistura natural de ouro e prata e foi feita há mais de 2.700 anos. \footnote{De acordo com o historiador grego Heródoto, que escreveu no século V AC, os lídios foram os primeiros a usaram moedas de ouro e prata. \cite{coinage-origins}} Se algo é novo no Bitcoin, o conceito de moeda não é isso.

\begin{figure}
  \centering
  \includegraphics[width=5cm]{assets/images/lydian-coin-stater.png}
  \caption{Uma moeda da Lídia. Imagem sob licença cc-by-sa da Classical Numismatic Group, Inc.}
  \label{fig:lydian-coin-stater}
\end{figure}

Acontece que acumular moedas, ou hodling, para usar o jargão atual, é quase tão antigo quanto as moedas da antiguidade. O primeiro criador de moedas foi alguém que colocou quase uma centena dessas moedas em um pote e as enterrou nas fundações de um templo, apenas para ser encontrado 2500 anos depois. Um armazenamento frio (também conhecido pelo jargão inglês de Cold Storage) muito bom, se você me perguntar.

Uma das desvantagens de se usar as moedas de metal precioso é que elas podem ser cortadas, efetivamente degradando o valor da moeda. Novas moedas podem ser cunhadas a partir dos pedaços, inflando a oferta de dinheiro ao longo do tempo, desvalorizando cada moeda individual no processo. As pessoas estavam literalmente cortando o máximo que podiam das suas moedas de dólares de prata. Eu me pergunto que tipo de anúncio de \textit{Clube das Tesouras} eles tinham naquela época.

Uma vez que os governos só ficam calmos com a inflação se são eles que a praticam, esforços foram feitos para impedir essa depreciação pelos cidadãos. No clássico estilo de polícia e ladrão, os depreciadores de moedas ficaram cada vez mais criativos com suas técnicas, forçando os \enquote{mestres da cunhagem} a ficarem ainda mais criativos em suas contra-medidas. Isaac Newton, o mundialmente conhecido físico que escreveu o livro \textit{Principia Mathematica}, costumava ser um desses mestres. Ele quem adicionou pequenas listras ao lado das moedas que ainda estão presentes hoje. Já se foram os dias em que era fácil depreciar as moedas.

\begin{figure}
  \includegraphics{assets/images/clipped-coins.png}
  \caption{Moedas de prata com cortes de tamanhos variados.}
  \label{fig:clipped-coins}
\end{figure}

Mesmo com esses métodos de depreciação das moedas \footnote{Além de recortar, dissolver (sacudir as moedas em um saco e coletar o seu pó) e tamponar (retirar o miolo da moeda fazendo um furo e depois, martelar a moeda até fechá-lo) foram os métodos mais utilizados para depreciar as moedas. \cite{wiki:coin-debasement}} mantidos sob controle, as moedas ainda sofrem de outros problemas. Elas são volumosas e não são muito convenientes para transportar, especialmente quando grandes transferências de valor precisam acontecer. Aparecer com uma enorme sacola de dólares de prata toda vez que você quiser comprar uma Mercedes não é algo muito prático.

Falando de coisas alemãs: como os \textit{dólares} dos Estados Unidos ganharam esse nome é outra história interessante. A palavra \enquote{dólar} é derivada da palavra alemã \textit{Thaler}, abreviação de \textit{Joachimsthaler}~\cite{wiki:thaler}. Um Joachimsthaler foi uma moeda cunhada na cidade de \textit{Sankt Joachimsthal}. O Thaler é simplesmente uma abreviatura para alguém (ou algo) vindo do vale, e porque Joachimsthal era \textit{o} vale da produção de moedas de prata, as pessoas simplesmente se referiam a essas moedas de prata como \textit{Thaler}. O Thaler (alemão) se transformou em daalders (holandês) e, finalmente, dólares (inglês).

\begin{figure}
  \centering
  \includegraphics[width=5cm]{assets/images/joachimsthaler.png}
  \caption{O 'Dólar' original. Saint Joachim foi colocado na face da mooeda com seu manto e seu chapéu de mago. A imagem está sob licença cc-by-sa da Wikipedia enviada pelo usuário Berlin-George}
  \label{fig:joachimsthaler}
\end{figure}

A introdução do dinheiro representativo foi o prenúncio da queda do dinheiro forte. Os certificados de ouro foram introduzidos em 1863 e, cerca de quinze anos depois, o dólar de prata também foi lenta, mas constantemente, sendo substituído por uma procuração em papel: o certificado de prata. \cite{wiki:silver-certificate}

Demorou cerca de 50 anos desde a introdução dos primeiros certificados de prata até que esses pedaços de papel se transformassem em algo que hoje reconheceríamos como um dólar americano.

\begin{figure}
  \centering
  \includegraphics{assets/images/us-silver-dollar-note-smaller.png}
  \caption{Um dólar americano de 1928. `Pagável ao portador sob demanda.' A imagem está sob licença cc-by-sa publicada pela National Numismatic Collection at the Smithsonian Institution}
  \label{fig:us-silver-dollar-note-smaller}
\end{figure}

Observe que o dólar de prata dos EUA de 1928 na Figura ~\ref{fig:us-silver-dollar-note-smaller} ainda atende pelo nome de \textit{certificado de prata}, indicando que este é de fato simplesmente um documento afirmando que o portador deste pedaço de papel tem direito a uma moeda de prata. É interessante ver que o texto que indica isso foi ficando menor com o tempo. O traço do \enquote{certificado} desapareceu completamente depois de um tempo, sendo substituído pela declaração tranquilizadora de que essas são notas do Federal Reserve.

Como mencionado acima, o mesmo aconteceu com o ouro. A maior parte do mundo seguia um padrão bimetálico ~\cite {wiki:bimetallism}, o que significa que as moedas eram feitas principalmente de ouro e prata. Ter certificados de ouro, resgatáveis em moedas de ouro, era indiscutivelmente uma melhoria tecnológica. O papel é mais conveniente, mais leve e, como pode ser dividido arbitrariamente, simplesmente imprimindo um número menor nele, é mais fácil dividi-lo em unidades menores.

Para lembrar aos portadores (usuários) que esses certificados eram representativos do ouro e da prata reais, eles tinham cores de acordo com o metal e isso era declarado claramente no próprio certificado. Você pode ler com facilidade a escrita de cima para baixo:

\begin{quotation}\begin{samepage}
\enquote{Isso certifica que foram depositados no tesouro dos Estados Unidos da América cem dólares em moedas de ouro pagáveis ao portador à vista.}
\end{samepage}\end{quotation}

\begin{figure}
  \centering
  \includegraphics{assets/images/us-gold-cert-100-smaller.png}
  \caption{Uma note de \$100 dólares cunhada em 1928 em certificados de ouro. A imagem está sob licença cc-by-sa publicada pela National Numismatic Collection, National Museum of American History.}
  \label{fig:us-gold-cert-100-smaller}
\end{figure}

Em 1963, as palavras \enquote{APAGAR AO PORTADOR SOB DEMANDA} foram removidas de todas as notas recém-emitidas. Cinco anos depois, o resgate das notas de papel por ouro e prata terminou.

As palavras que sugeriam as origens e a ideia por trás do papel-moeda foram removidas. A cor dourada desapareceu. Tudo o que restou foi o papel e com ele a capacidade do governo de imprimir o quanto quiser.

Com a abolição do padrão-ouro em 1971, esse truque de prestidigitação secular estava finalizado. O dinheiro se tornou a ilusão que todos compartilhamos até hoje: um dinheiro fiduciário. Vale alguma coisa porque alguém comandando um exército e operando prisões diz que vale alguma coisa. Como pode ser lido claramente em cada nota de dólar em circulação hoje, \enquote{ESTA NOTA É DE CURSO LEGAL}. Em outras palavras: é valiosa porque a nota diz que ela é valiosa.

\begin{figure}
  \centering
  \includegraphics{assets/images/us-dollar-2004.jpg}
  \caption{Uma nota de vinte dólares americanos de 2004 usada atualmente. `ESTA NOTA É DE CURSO LEGAL'}
  \label{fig:us-dollar-2004}
\end{figure}

A propósito, há outra lição interessante sobre as notas de hoje, que está escondido mas ao mesmo tempo estampado na cara de todo mundo. A segunda linha diz que ela é de curso legal \enquote{PARA TODAS AS DÍVIDAS, PÚBLICAS E PRIVADAS}. O que pode ser óbvio para os economistas, me surpreendeu: todo dinheiro é dívida. Minha cabeça ainda está doendo por causa disso, e deixarei a exploração da relação entre dinheiro e dívida como um exercício para você, leitor.

\paragraph{}
Como vimos, ouro e prata foram usados como dinheiro por milênios. Com o tempo, as moedas feitas de ouro e prata foram substituídas por papel. O papel foi lentamente sendo aceito como forma de pagamento. Essa aceitação criou uma ilusão - a ilusão de que o próprio papel tem valor. O movimento final foi cortar completamente o vínculo entre a representação e o seu valor real: abolindo o padrão-ouro e convencendo a todos de que o papel em si é precioso.

\paragraph{O Bitcoin me ensinou sobre a história do dinheiro e o maior truque da história da economia: a moeda fiduciária.}

% ---
%
% #### Down the Rabbit Hole
%
% - [Shelling Out: The Origins of Money] by Nick Szabo
% - [Methods of Coin Debasement][coin debasement], [Thaler], [U.S. Silver Certificate][silver certificates], [Bimetallism][bimetallic standard] on Wikipedia
%
% [oldest coin]: https://www.britishmuseum.org/explore/themes/money/the_origins_of_coinage.aspx
% [coin debasement]: https://en.wikipedia.org/wiki/Methods_of_coin_debasement
% [Thaler]: https://en.wikipedia.org/wiki/Thaler
% [Berlin-George]: https://en.wikipedia.org/wiki/File:Bohemia,_Joachimsthaler_1525_Electrotype_Copy._VF._Obverse..jpg
% [silver certificates]: https://en.wikipedia.org/wiki/Silver_certificate_%28United_States%29
% [bimetallic standard]: https://en.wikipedia.org/wiki/Bimetallism
% [Shelling Out: The Origins of Money]: https://nakamotoinstitute.org/shelling-out/
%
% <!-- Wikipedia -->
% [alice]: https://en.wikipedia.org/wiki/Alice%27s_Adventures_in_Wonderland
% [carroll]: https://en.wikipedia.org/wiki/Lewis_Carroll

\input{lessons/ch2-13-fractional-reserve-insanity}
\input{lessons/ch2-14-sound-money}
\part{Tecnologia}
\label{ch:technology}
\chapter*{Tecnologia}

\begin{chapquote}{Lewis Carroll, \textit{Alice no País das Maravilhas}}
\enquote{Desta vez vou me sair melhor}, disse para si mesma, e começou por pegar a chavezinha de ouro e destrancar a porta que dava para o jardim.
\end{chapquote}

Chaves de ouro, relógios que só funcionam por acaso, corridas para resolver enigmas estranhos e construtores que não têm rostos nem nomes. O que parece contos de fadas do País das Maravilhas é um negócio comum no mundo do Bitcoin.

Como exploramos no Capítulo~\ref{ch:economics}, grandes partes do sistema financeiro atual são sistematicamente falhas. Como Alice, só podemos esperar administrar melhor desta vez. Mas, graças a um inventor pseudônimo, temos uma tecnologia incrivelmente sofisticada para nos apoiar desta vez: o Bitcoin.

Resolver problemas em um ambiente radicalmente descentralizado e adversário requer soluções únicas. O que de outra forma seriam problemas triviais para resolver são tudo, menos isso, neste mundo estranho. O Bitcoin depende de uma forte criptografia para a maioria das soluções, pelo menos quando analisado através das lentes da tecnologia. Iremos explorar o quão forte essa criptografia é, em uma das lições a seguir.

A criptografia é o que o Bitcoin usa para remover a confiança nas autoridades. Ao invés de depender de instituições centralizadas, o sistema depende da autoridade final do nosso universo: a física. Alguns pequenos grãos de confiança ainda são necessários, no entanto. Examinaremos esses grãos na segunda lição deste capítulo.

~

\begin{samepage}
Parte~\ref{ch:technology} -- Tecnologia:

\begin{enumerate}
  \setcounter{enumi}{14}
  \item Força nos números
  \item Reflexos no \enquote{Não Confie, Verifique}
  \item Dizer o tempo demanda trabalho
  \item Mova-se lentamente e não quebre as coisas
  \item A Privacidade não morreu
  \item Cypherpunks escrevem códigos
  \item Metáforas para um futuro do Bitcoin
\end{enumerate}
\end{samepage}

As últimas duas lições exploram o \textit{ethos} do desenvolvimento tecnológico no Bitcoin, que é indiscutivelmente tão importante quanto a própria tecnologia. O Bitcoin não é o próximo aplicativo revolucionário no seu celular. É a base de uma nova realidade econômica, razão pela qual o Bitcoin deve ser tratado como um software financeiro de nível nuclear.

Onde estamos nesta revolução financeira, social e tecnológica? Redes e tecnologias do passado podem servir como metáforas para o futuro dos Bitcoins, que são exploradas na última lição deste capítulo.

Mais uma vez, aperte o cinto e aproveite o passeio. Como todas as tecnologias exponenciais, estamos prestes a nos tornar parabólicos.
\input{lessons/ch3-15-strength-in-numbers}
\chapter{Reflexos do \enquote{Não Confie, Verifique}}
\label{les:16}

\begin{chapquote}{Lewis Carroll, \textit{Alice no País das Maravilhas}}
\enquote{Agora, para as evidências}, disse o Rei, \enquote{e depois para a sentença.\footnote{Nota do tradutor: Esse trecho da Alice no País das Maravilhas não foi encontrada em nenhuma das edições utilizadas na tradução, por isso foi feita a tradução do texto incluído pelo autor.}}
\end{chapquote}

O Bitcoin visa substituir, ou pelo menos fornecer uma alternativa à moeda convencional. A moeda convencional está vinculada a uma autoridade centralizada, não importa se estamos falando de moeda corrente como o dólar americano ou dinheiro de monopólio moderno como os V-Bucks do Fortnite. Em ambos os exemplos, você deve confiar na autoridade central para emitir, gerenciar e distribuir seu dinheiro. O Bitcoin acaba com essa necessidade, e o principal problema que o Bitcoin resolve é o problema de \textit{confiança}.

\begin{quotation}\begin{samepage}
\enquote{A raiz do problema com a moeda convencional é toda a confiança necessária para fazê-la funcionar. [...] o que é necessário é um sistema de pagamento eletrônico baseado em prova criptográfica ao invés da confiança.}
\begin{flushright} -- Satoshi Nakamoto\footnote{Satoshi Nakamoto, anúncio oficial do Bitcoin~\cite{bitcoin-announcement} e no whitepaper~\cite{whitepaper}}
\end{flushright}\end{samepage}\end{quotation}

O Bitcoin resolve o problema de confiança por ser completamente descentralizado, sem a necessidade de um servidor central ou de partes confiáveis. Nem mesmo é necessário \textit{terceiros} confiáveis, mas partes confiáveis, ponto final. Quando não há autoridade central, simplesmente \textit{não} há ninguém em quem confiar. A descentralização completa é a inovação. É a raiz da resiliência do Bitcoin, a razão pela qual ele ainda está vivo. A descentralização também é o motivo pelo qual temos a mineração, nodes, carteiras físicas e, sim,a blockchain. A única coisa que você tem que \enquote{confiar} é que nosso entendimento de matemática e física não está totalmente errado e que a maioria dos mineradores age honestamente (o que eles são incentivados a fazer).

Enquanto o mundo normal opera sob o pressuposto de \textit{\enquote {confie, mas verifique}}, o Bitcoin opera sob o pressuposto de \textit{\enquote{não confie, verifique}}. Satoshi destacou a importância de remover a confiança, tanto na introdução, quanto na conclusão do whitepaper do Bitcoin.

\begin{quotation}\begin{samepage}
\enquote{Conclusão: Propomos um sistema para transações eletrônicas sem dependência da confiança.}
\begin{flushright} -- Satoshi Nakamoto\footnote{Satoshi Nakamoto, the Bitcoin whitepaper~\cite{whitepaper}}
\end{flushright}\end{samepage}\end{quotation}

Observe que \textit{sem dependência da confiança} é usado em um contexto muito específico aqui. Estamos falando de terceiros confiáveis, ou seja, outras entidades nas quais você confia para produzir, manter e processar seu dinheiro. Presume-se, por exemplo, que você pode confiar no seu computador.

Como Ken Thompson mostrou na palestra no Prêmio Turing, confiança é uma coisa extremamente complicada no mundo computacional. Ao executar um programa, você deve confiar em todos os tipos de software (e hardware) que, em teoria, podem alterar o programa que você está tentando executar de forma maliciosa. Como Thompson resumiu em seu livro \textit{Reflexões sobre a necessidade de confiar na confiança}: \enquote{A moral é óbvia. Você não pode confiar em um código que não foi totalmente criado por você mesmo.}~\cite{trusting-trust}

\begin{figure}
  \includegraphics{assets/images/ken-thompson-hack.png}
  \caption{Trechos do artigo de Ken Thompson 'Reflexões sobre confiar na confiança'}
  \label{fig:ken-thompson-hack}
\end{figure}

Thompson demonstrou que mesmo se você tiver acesso ao código-fonte, seu compilador - ou qualquer outro programa ou hardware de gerenciamento de programa - pode estar comprometido e detectar esse backdoor seria muito difícil. Assim, na prática, um sistema verdadeiramente \textit{confiável} não existe. Você teria que criar todo o seu software \textit{e} todo o seu hardware (montadores, compiladores, vinculadores, etc.) a partir do zero, sem a ajuda de nenhum software externo ou maquinário auxiliado por software.

\begin{quotation}\begin{samepage}
\enquote{Se você deseja fazer uma torta de maçã do zero, você deve primeiro inventar o universo.}
\begin{flushright} -- Carl Sagan\footnote{Carl Sagan, \textit{Cosmos} \cite{cosmos}}
\end{flushright}\end{samepage}\end{quotation}

O hack do Ken Thompson é um backdoor particularmente engenhoso e difícil de detectar, então vamos dar uma olhada rápida em um backdoor difícil de ser detectado, que funciona sem modificar nenhum software. Os pesquisadores descobriram uma maneira de comprometer o hardware crítico de segurança alterando a polaridade das impurezas do silício.~\cite{becker2013stealthy} Apenas mudando as propriedades físicas das coisas de que os chips de computador são feitos, eles foram capazes de comprometer um gerador de números aleatórios criptograficamente seguro. Como essa mudança não pode ser encontrada, o backdoor não pode ser detectado por inspeção óptica, que é um dos mecanismos de detecção de violação mais importantes para chips como esses.

\begin{figure}
  \includegraphics{assets/images/stealthy-hardware-trojan.png}
  \caption{Os Cavalos de Tróia Dopant-Level Escondidos no Hardware por Becker, Regazzoni, Paar, Burleson}
  \label{fig:stealthy-hardware-trojan}
\end{figure}

Parece assustador? Bem, mesmo se você fosse capaz de construir tudo do zero, ainda teria que confiar na matemática. Você teria que confiar que \textit{secp256k1} é uma curva elíptica que não possui nenhum tipo de backdoor ou erro. Sim, backdoors maliciosos podem ser inseridos nas bases matemáticas das funções criptográficas e, sem dúvida, isso já aconteceu pelo menos uma vez.~\cite{wiki:Dual_EC_DRBG} Existem boas razões para você ser paranóico e o fato de que tudo, desde o seu hardware, até seu software, passando pelas curvas elípticas utilizadas podem ter backdoors~\cite{wiki:backdoors}, são alguns dos motivos.

\begin{quotation}\begin{samepage}
\enquote{Não confie. Verifique.}
\begin{flushright} -- Bitcoinheiros de todos os lugares
\end{flushright}\end{samepage}\end{quotation}

Os exemplos acima devem ilustrar que a computação \textit{confiável} é utópica. O Bitcoin é provavelmente o sistema que mais se aproxima dessa utopia, mas ainda assim, é \textit{preciso um certo mínimo de confiança} --- com o objetivo de remover a confiança sempre que possível. Indiscutivelmente, a cadeia de confiança é interminável, já que você também terá que confiar que a computação requer energia, que P não é igual a NP e que você está realmente na realidade básica e não preso em uma simulação por agentes mal-intencionados.

Os desenvolvedores estão trabalhando em ferramentas e procedimentos para minimizar ainda mais qualquer confiança remanescente. Por exemplo, os desenvolvedores do Bitcoin criaram o Gitian \footnote{\url{https://gitian.org/}}, que é um método de distribuição de software para criar construções determinísticas. A ideia é que, se vários desenvolvedores forem capazes de reproduzir binários idênticos, a chance de adulteração maliciosa será reduzida. Os backdoors mais fantasiosos não são o único vetor de ataque. A simples chantagem ou extorsão também são ameaças reais. Como no protocolo principal, a descentralização é usada para minimizar a confiança.

Vários esforços estão sendo feitos para melhorar o problema do ovo e da galinha do bootstrapping, que o hack de Ken Thompson tão brilhantemente apontou~\cite{web:bootstrapping}. Um desses esforços é o Guix\footnote{\url{https://guix.gnu.org}} (a pronuncia correta é \textit{geeks}), que usa gerenciamento de pacote declarado funcionalmente levando a compilações reproduzíveis bit a bit por design. O resultado é que você não precisa mais confiar em nenhum servidor que fornece o software, pois pode verificar se o binário fornecido não foi adulterado, reconstruindo-o do zero. Recentemente, um pull request foi mesclado para integrar o Guix ao processo de build do Bitcoin.\footnote{Veja o PR 15277 do \texttt{bitcoin-core}: \url{https://github.com/bitcoin/bitcoin/pull/15277}}

\begin{figure}
  \includegraphics{assets/images/guix-bootstrap-dependencies.png}
  \caption{O que veio primeiro, o ovo ou a galinha?}
  \label{fig:guix-bootstrap-dependencies}
\end{figure}

Felizmente, o Bitcoin não depende de um único algoritmo ou peça de hardware. Um efeito da descentralização radical do Bitcoin é um modelo de segurança distribuído. Embora as backdoors descritos acima não devam ser desconsiderados, é improvável que cada software de carteira, cada carteira física, cada biblioteca criptográfica, cada implementação de node e cada compilador de cada linguagem sejam comprometidos. Possível, mas altamente improvável.

Observe que você pode gerar uma chave privada sem depender de nenhum hardware ou software computacional. Você pode jogar uma moeda ~\cite{antonopoulos2014mastering} algumas vezes, embora dependendo de sua moeda e estilo de lançamento esta fonte de aleatoriedade possa não ser suficientemente aleatória. Há um motivo pelo qual protocolos de armazenamento como Glacier\footnote{\url{https://glacierprotocol.org/}} aconselham o uso de dados de nível de cassino como uma das duas fontes de entropia.

O Bitcoin me forçou a refletir sobre o que realmente significa não confiar em ninguém. Isso aumentou minha consciência do problema do bootstrap e da cadeia de confiança implícita no desenvolvimento e execução de software. Isso também aumentou minha consciência sobre as muitas maneiras pelas quais software e hardware podem ser comprometidos.

\paragraph{O Bitcoin me ensinou a não confiar, mas a verificar.}

% ---
%
% #### Down the Rabbit Hole
%
% - [The Bitcoin whitepaper][Nakamoto] by Satoshi Nakamoto
% - [Reflections on Trusting Trust][\textit{Reflections on Trusting Trust}] by Ken Thompson
% - [51% Attack][majority] on the Bitcoin Developer Guide
% - [Bootstrapping][bootstrapping], Guix Manual
% - [Secp256k1][secp256k1] on the Bitcoin Wiki
% - [ECC Backdoors][backdoors], [Dual EC DRBG][has already happened] on Wikipedia
%
% [Emmanuel Boutet]: https://commons.wikimedia.org/wiki/User:Emmanuel.boutet
% [\textit{Reflections on Trusting Trust}]: https://www.archive.ece.cmu.edu/~ganger/712.fall02/papers/p761-thompson.pdf
% [found a way]: https://scholar.google.com/scholar?hl=en&as_sdt=0%2C5&q=Stealthy+Dopant-Level+Hardware+Trojans&btnG=
% [Gitian]: https://gitian.org/
% [bootstrapping]: https://www.gnu.org/software/guix/manual/en/html_node/Bootstrapping.html
% [Guix]: https://www.gnu.org/software/guix/
% [pull-request]: https://github.com/bitcoin/bitcoin/pull/15277
% [flip a coin]: https://github.com/bitcoinbook/bitcoinbook/blob/develop/ch04.asciidoc#private-keys
% [Glacier]: https://glacierprotocol.org/
% [secp256k1]: https://en.bitcoin.it/wiki/Secp256k1
% [majority]: https://bitcoin.org/en/developer-guide#term-51-attack
%
% <!-- Wikipedia -->
% [backdoors]: https://en.wikipedia.org/wiki/Elliptic-curve_cryptography#Backdoors
% [has already happened]: https://en.wikipedia.org/wiki/Dual_EC_DRBG
% [Carl Sagan]: https://en.wikipedia.org/wiki/Cosmos_%28Carl_Sagan_book%29
% [alice]: https://en.wikipedia.org/wiki/Alice%27s_Adventures_in_Wonderland
% [carroll]: https://en.wikipedia.org/wiki/Lewis_Carroll

\input{lessons/ch3-17-telling-time-takes-work}
\chapter{Mova-se lentamente e não quebre as coisas}
\label{les:18}

\begin{chapquote}{Lewis Carroll, \textit{Alice no País das Maravilhas}}
Assim, o barco navegou lentamente, sob o brilhante dia de verão, com sua tripulação alegre e sua música de vozes e risos\footnote{Nota do tradutor: Esse trecho da Alice no País das Maravilhas não foi encontrada em nenhuma das edições utilizadas na tradução, por isso foi feita a tradução do texto incluído pelo autor.}\ldots
\end{chapquote}

Pode ser um mantra esquecido, mas \enquote{move-se rápido e quebrar as coisas} ainda é o \textit{modus operandi} da tecnologia contemporânea. A ideia de que, não importa se você acertar tudo na primeira vez, é um pilar básico da mentalidade \textit{falhe cedo, falhe frequentemente}. O sucesso é medido pelo crescimento, então, enquanto você está crescendo, tudo bem. Se algo não funcionar no início, você simplesmente faz o pivoteamento e itera. Em outras palavras: jogue merda no ventilador e veja qual que gruda.

O Bitcoin é muito diferente. É diferente por design. É diferente por necessidade. Como Satoshi apontou, a moeda eletrônica já foi tentada muitas vezes anteriormente, e todas as tentativas falharam porque os desenvolvedores criavam uma fera que tinha uma cabeça para ser cortada. A novidade do Bitcoin, é que ele é uma besta sem cabeça.

\begin{quotation}\begin{samepage}
\enquote{Muitas pessoas descartam automaticamente a moeda eletrônica como uma causa perdida
por conta de todas as empresas que faliram desde a década de 1990. Espero que seja
óbvio que foi apenas a natureza centralizada dos sistemas que fizeram com que elas estivessem condenadas.}
\begin{flushright} -- Satoshi Nakamoto\footnote{Satoshi Nakamoto, em resposta ao usuário Sepp Hasslberger. \cite{satoshi-centralized-nature}}
\end{flushright}\end{samepage}\end{quotation}

Uma consequência dessa descentralização radical é uma resistência inerente à mudança. \enquote{Mova-se rápido e quebre as coisas} não funciona e nunca funcionará na camada base do Bitcoin. Mesmo que fosse desejável, não seria possível convencer \textit{todos} os usuários a mudarem seus hábitos. Isso é consenso distribuído. Essa é a natureza do Bitcoin.

\begin{quotation}\begin{samepage}
\enquote{A natureza do Bitcoin é tal que, uma vez que a versão 0.1 foi lançada, o projeto principal foi gravado em pedra para o resto de sua vida.}
\begin{flushright} -- Satoshi Nakamoto\footnote{Satoshi Nakamoto, em resposta ao usuário Gavin Andresen \cite{satoshi-centralized-nature}}
\end{flushright}\end{samepage}\end{quotation}

Esta é uma das muitas propriedades paradoxais do Bitcoin. Todos nós acreditamos que qualquer coisa que seja software pode ser alterada facilmente. Mas a natureza da besta torna muito difícil mudá-la.

Como Hasu mostra lindamente em Abrindo o Contrato Social do Bitcoin~\cite{contrato-social}, mudar as regras do Bitcoin só é possível \textit{propondo} uma mudança e, consequentemente, \textit{convencendo} todos os usuários do Bitcoin a adotarem essa mudança. Isso torna o Bitcoin muito resistente a alterações, mesmo sendo um software.

Essa resiliência é uma das propriedades mais importantes do Bitcoin. Os sistemas de software críticos têm que ser antifrágeis. É isso que a interação da camada social do Bitcoin e sua camada técnica garantem. Os sistemas monetários são adversários por natureza e, como sabemos há milhares de anos, bases sólidas são essenciais em um ambiente hostil.

\begin{quotation}\begin{samepage}
\enquote{E desceu a chuva, e correram rios, e assopraram ventos, e combateram aquela casa, e não caiu, porque estava edificada sobre a rocha.}
\begin{flushright} -- Matheus 7:24--27
\end{flushright}\end{samepage}\end{quotation}

Indiscutivelmente, nesta parábola dos construtores sábios e tolos, o Bitcoin não é a casa. É a rocha. Imutável, imóvel, fornecendo a base para um novo sistema financeiro.

Assim como os geólogos, que sabem que as formações rochosas estão sempre se movendo e evoluindo, pode-se ver que o Bitcoin está sempre se movendo e evoluindo também. Você só precisa saber para onde olhar e como olhar para ele.

A introdução de pay to script hash \footnote{Transações do tipo Pay to script hash (P2SH) foram padronizadas no BIP16. Eles permitem que as transações sejam enviadas para um script hash (endereço começando com 3) ao invés de um hash de chave pública (endereços começando com 1) ~\cite{btcwiki:p2sh}} e segregated witnesses\footnote{Segregated Witness (abreviado como SegWit) é uma atualização de protocolo implementada com o objetivo de fornecer proteção contra maleabilidade de transação além de aumentar a capacidade do bloco. O SegWit separa a \textit{testemunha} da lista de entradas.~\cite{btcwiki:segwit}} São a prova de que as regras do Bitcoin podem ser alteradas se um número suficiente de usuários estiver convencido de que adotar tal alteração é benéfico para a rede. Este último possibilitou o desenvolvimento da rede lightning\footnote{\url{https://lightning.network/}}, que é uma das casas que estão sendo construídas sobre a base sólida do Bitcoin. Atualizações futuras como assinaturas Schnorr~\cite{bip:schnorr} irão aumentar a eficiência e privacidade, bem como scripts (leia-se: contratos inteligentes) que serão indistinguíveis de transações regulares graças ao Taproot~\cite{taproot}. Construtores sábios realmente constroem em bases sólidas.

O Satoshi não era apenas um construtor tecnologicamente sábio. Ele também entendeu que seria necessário tomar decisões acertadas ideologicamente.

\begin{quotation}\begin{samepage}
\enquote{Ser código aberto significa que qualquer pessoa pode revisar o código de forma independente. Se fosse de código fechado, ninguém poderia verificar a segurança. Eu acho que é essencial para um programa desta natureza que seu código seja aberto.}
\begin{flushright} -- Satoshi Nakamoto\footnote{Satoshi Nakamoto, em resposta ao usuário SmokeTooMuch \cite{satoshi-open-source}}
\end{flushright}\end{samepage}\end{quotation}

A abertura é fundamental para a segurança e inerente ao código aberto e ao movimento do software livre. Como Satoshi apontou, os protocolos seguros e o código que os implementa devem ser abertos --- não há segurança através da obscuridade. Outro benefício está novamente relacionado à descentralização: o código que pode ser executado, estudado, modificado, copiado e distribuído gratuitamente garante que ele seja espalhado por toda parte.

A natureza radicalmente descentralizada do Bitcoin é o que o faz se mover lenta e progressivamente. Uma rede de nodes, cada um administrado por um indivíduo soberano, é inerentemente resistente a mudanças - maliciosas ou não. Sem nenhuma maneira de forçar as atualizações aos usuários, a única maneira de introduzir mudanças é convencer lentamente cada um desses indivíduos a adotá-la. Esse processo descentralizado de introdução e implantação de alterações é o que torna a rede incrivelmente resistente a mudanças maliciosas. É também o que torna mais difícil consertar coisas quebradas do que em um ambiente centralizado, razão pela qual ninguém quebra nada em primeiro lugar.

\paragraph{O Bitcoin me ensinou que mover-se devagar é uma de suas características, não um bug.}

% ---
%
% #### Through the Looking-Glass
%
% - [Lesson 1: Immutability and Change][lesson1]
%
% #### Down the Rabbit Hole
%
% - [Unpacking Bitcoin's Social Contract] by Hasu
% - [Schnorr signatures BIP][Schnorr signatures] by Pieter Wuille
% - [Taproot proposal][Taproot] by Gregory Maxwell
% - [P2SH][pay to script hash], [SegWit][segregated witness] on the Bitcoin Wiki
% - [Parable of the Wise and the Foolish Builders][Matthew 7:24--27] on Wikipedia
%
% <!-- Down the Rabbit Hole -->
% [lesson1]: {{ '/bitcoin/lessons/ch1-01-immutability-and-change' | absolute_url }}
%
% [Unpacking Bitcoin's Social Contract]: https://uncommoncore.co/unpacking-bitcoins-social-contract/
% [Matthew 7:24--27]: https://en.wikipedia.org/wiki/Parable_of_the_Wise_and_the_Foolish_Builders
% [pay to script hash]: https://en.bitcoin.it/wiki/Pay_to_script_hash
% [segregated witness]: https://en.bitcoin.it/wiki/Segregated_Witness
% [lightning network]: https://lightning.network/
% [Schnorr signatures]: https://github.com/sipa/bips/blob/bip-schnorr/bip-schnorr.mediawiki#cite_ref-6-0
% [Taproot]: https://lists.linuxfoundation.org/pipermail/bitcoin-dev/2018-January/015614.html
%
% <!-- Wikipedia -->
% [alice]: https://en.wikipedia.org/wiki/Alice%27s_Adventures_in_Wonderland
% [carroll]: https://en.wikipedia.org/wiki/Lewis_Carroll

\chapter{Privacidade Não Morreu}
\label{les:19}

\begin{chapquote}{Lewis Carroll, \textit{Alice no País das Maravilhas}}
	Todos os jogadores jogaram juntos, sem esperar por turno, e discutiram 
	gritando a plenos pulmões, e em poucos minutos Rainha, em uma paixão 
	furiosa, batendo os pés e gritando \enquote{cortem a cabeça dele!} e 
	\enquote{cortem a cabeça dela!} mais ou menos a cada minuto.
\end{chapquote}

Se você acreditar nos especialistas, a privacidade morreu desde os anos 80
\footnote{\url{https://bit.ly/privacy-is-dead}}. A invenção do Bitcoin por um 
pseudônimo e outros eventos na história corrente mostram que isso não é verdade.
Privacidade está viva, mesmo que não seja fácil escapar de um Estado de vigilância. 

Satoshi fez um grande esforço para encobrir seus rastros e esconder
sua identidade. Dez anos depois, ainda não se sabe se Satoshi Nakamoto era uma pessoa 
solteira, um grupo de pessoas, homem, mulher ou uma IA que viaja no tempo que se 
autoinicializou a si mesmo para dominar o mundo. Teorias da conspiração à parte, Satoshi 
escolheu se identificar como um japonês, e é por isso que eu não suponho, mas respeito seu 
gênero escolhido e me refiro a ele como \textit{ele}. 

\begin{figure}
  \includegraphics{assets/images/nope.png}
  \caption{Eu não sou Dorian Nakamoto.}
  \label{fig:nope}
\end{figure}

Qualquer que seja a sua identidade real, Satoshi teve sucesso em escondê-la. 
Ele deu um exemplo encorajador para todo mundo que deseja ficar anônimo: 
é possível ter privacidade online.

\begin{quotation}\begin{samepage}
\enquote{Encriptação Funciona. Um sistema de criptografia propriamente implementado é 
	uma das poucas coisas em que se pode confiar.}
\begin{flushright} -- Edward Snowden\footnote{Edward Snowden, respondendo à perguntas dos leitores \cite{snowden}}
\end{flushright}\end{samepage}\end{quotation}

Satoshi não foi o primeiro inventor anônimo ou pseudônimo, e ele não vai ser o último.
Alguns imitaram seu estilo de publicação pseudônima, como Tom Elvis Yedusor, criador da 
MimbleWimble~\cite{mimblewimble-origin} enquanto outros publicaram provas matemáticas avançadas 
e permaneceram completamente anônimos~\cite{4chan-math}.

Esse mundo em que vivemos é estranho. Um mundo onde identidade é opcional, 
contribuições são aceitas baseadas no mérito, e pessoas podem colaborar e transacionar livremente.
Vai demorar ainda algum tempo para me ajustar e ficar confortável com esses paradigmas, mas eu acredito 
fortemente que todas essas coisas tem potencial de mudar o mundo para melhor.

Nós sempre devemos lembrar que privacidade é um direito humano\footnote{Universal 
	Declaration of Human Rights, \textit{Article 12}.~\cite{article12}} fundamental. 
E enquanto as pessoas exercerem e defenderem 
esses direitos, a batalha por privacidade está longe de acabar.

\paragraph{Bitcoin me ensinou que a privacidade não morreu.}

% ---
%
% #### Down the Rabbit Hole
%
% - [Universal Declaration of Human Rights][fundamental human right] by the United Nations
% - [A lower bound on the length of the shortest superpattern][anonymous] by Anonymous 4chan Poster, Robin Houston, Jay Pantone, and Vince Vatter
%
% [since the 80ies]: https://books.google.com/ngrams/graph?content=privacy+is+dead&year_start=1970&year_end=2019&corpus=15&smoothing=3&share=&direct_url=t1%3B%2Cprivacy%20is%20dead%3B%2Cc0
% [time-traveling AI]: https://blockchain24-7.com/is-crypto-creator-a-time-travelling-ai/
% ["I am not Dorian Nakamoto."]: http://p2pfoundation.ning.com/forum/topics/bitcoin-open-source?commentId=2003008%3AComment%3A52186
% [MimbleWimble]: https://github.com/mimblewimble/docs/wiki/MimbleWimble-Origin
% [anonymous]: https://oeis.org/A180632/a180632.pdf
% [fundamental human right]: http://www.un.org/en/universal-declaration-human-rights/
%
% <!-- Wikipedia -->
% [alice]: https://en.wikipedia.org/wiki/Alice%27s_Adventures_in_Wonderland
% [carroll]: https://en.wikipedia.org/wiki/Lewis_Carroll

\chapter{Cypherpunks Escrevem Código}
\label{les:20}

\begin{chapquote}{Lewis Carroll, \textit{Alice no País das Maravilhas}}
\enquote{Eu posso ver que você está tentando inventar alguma coisa.}
\end{chapquote}

Grandes ideias não surgem do nada, com o Bitcoin, também foi assim. 
Ele só foi possível combinando muitas inovações e descobertas da matemática, 
física, ciências da computação, e muitas outras áreas. Mesmo sendo um gênio, 
Satoshi não conseguiria inventar o Bitcoin sem estar apoiado sobre os ombros de gigantes.

\begin{quotation}\begin{samepage}
\enquote{Aquele que apenas espera e deseja não interfere ativamente no curso dos eventos e com a criação do próprio destino.}
\begin{flushright} -- Ludwig von Mises\footnote{Ludwig von Mises, \textit{Ação Humana} \cite{human-action}}
\end{flushright}\end{samepage}\end{quotation}
% > <cite>[Ludwig Von Mises]</cite>

Um desses gigantes é o Eric Hughes, um dos fundadores do movimento cypherpunk
e escritor do \textit{O Manifesto dos Cypherpunks}. Em inglês \textit{A Cypherpunk's Manifesto}. 
É difícil imaginar que Satoshi não tenha sido influenciado por esse manifesto. Nele já se falava de muitas coisas que o Bitcoin proporciona e usa, como transações diretas e privadas, dinheiro e moedas eletrônicas, sistemas anônimos e, a defesa de privacidade com criptografia e assinaturas digitais.

\begin{quotation}\begin{samepage}
\enquote{Privacidade é necessária para uma sociedade livre na era eletrônica.
	[...] Já que desejamos privacidade, nós temos que ter certeza que cada parte 
	na transação tenha conhecimento só do que é diretamente necessário para 
	aquela transação [...]
	Assim, privacidade em uma sociedade livre requer sistemas de transação anônima. 
	Até agora, o dinheiro tem sido o principal sistema desse tipo.
	O sistema de transações anônimas não é um sistema de transações secretas. [...]
	Nós os Cypherpunks, estamos dedicados em construir sistemas anônimos. Nós vamos 
	defender a nossa privacidade com criptografia, com sistemas de encaminhamento 
	de e-mails anônimos, com assinaturas digitais, e com dinheiro eletrônico. 
	Os cypherpunks escrevem código.}

\begin{flushright} -- Eric Hughes\footnote{Eric Hughes, O Manifesto dos Cypherpunks \cite{cypherpunk-manifesto}}
\end{flushright}\end{samepage}\end{quotation}

Cypherpunks não se confortam com esperanças e desejos. 
Eles ativamente interferem com os eventos em curso e moldam seu próprio destino. 
Os cypherpunks escrevem código.

Assim, no verdadeiro estilo cypherpunk, Satoshi sentou e começou a escrever código. 
Código que de uma ideia abstrata e provou para o mundo que ela realmente funcionava. 
Código que plantou a semente de uma nova realidade econômica.
Graças a esse código, todo mundo pode verificar que esse sistema novo realmente funciona, 
a cada dez minutos mais ou menos, o Bitcoin provou isso para o mundo e ainda vive.

\begin{figure}
  \includegraphics{assets/images/bitcoin-code-white.png}
  \caption{Trechos de código do Bitcoin version 0.1}
  \label{fig:bitcoin-code-white}
\end{figure}

Para ter certeza de que sua inovação transcende a fantasia e vira realidade, Satoshi 
escreveu o código para implementar essa ideia antes de escrever o artigo técnico. 
Ele também fez questão de não atrasar\footnote{\enquote{Nós não devemos atrasar 
		se todos os recursos estão prontos.} 
	-- Satoshi Nakamoto~\cite{satoshi-delay}} qualquer versão para sempre.
	Até porque, \enquote{sempre tem mais uma coisa ser feita.}

\begin{quotation}\begin{samepage}
\enquote{Eu tive que escrever todo aquele código, antes de me convencer que eu podia resolver todo problema, só depois eu escrevi o artigo.}
\begin{flushright} -- Satoshi Nakamoto\footnote{Satoshi Nakamoto, em resposta ao Bitcoin P2P e-cash paper \cite{satoshi-mail-code-first}}
\end{flushright}\end{samepage}\end{quotation}

No mundo de hoje, cheio de promessas de execução duvidosa, um exercício de construção dedicada 
era desesperadamente necessário. Propositalmente, para convencer a si mesmo de que você pode realmente resolver os problemas e, implementar as soluções. 
Todos nós devemos ter como objetivo um pouco mais cypherpunk. 

\paragraph{O Bitcoin me ensinou que cypherpunks escrevem código.}

% ---
%
% #### Down the Rabbit Hole
%
% - [Bitcoin version 0.1.0 announcement][version 0.1.0] by Satoshi Nakamoto
% - [Bitcoin P2P e-cash paper announcement][mail-announcement] by Satoshi Nakamoto
%
% [mail-announcement]: http://www.metzdowd.com/pipermail/cryptography/2008-October/014810.html
% [Ludwig Von Mises]: https://mises.org/library/human-action-0/html/pp/613
% [version 0.1.0]: https://bitcointalk.org/index.php?topic=68121.0
% [not to delay]: https://bitcointalk.org/index.php?topic=199.msg1670#msg1670
% [6]: http://www.metzdowd.com/pipermail/cryptography/2008-November/014832.html
%
% <!-- Wikipedia -->
% [alice]: https://en.wikipedia.org/wiki/Alice%27s_Adventures_in_Wonderland
% [carroll]: https://en.wikipedia.org/wiki/Lewis_Carroll

\chapter{Metáfora para o futuro do Bitcoin}
\label{les:21}

\begin{chapquote}{Lewis Carroll, \textit{Alice no País das Maravilhas}}
\enquote{Eu sei que algo interessante vai certamente acontecer\ldots}
\end{chapquote}

Nas últimas decadas, a inovação tecnológica aparentemente não tem seguido 
uma tendência linear. Quer você acredite que singularidade tecnológica existe 
ou não, é inegável o progresso exponencial em muitas áreas. Além do fato da velocidade 
em que as tecnologias estão sendo adotadas estar acelerando, e antes 
que você perceba, o playground do pátio do maternal desapareceu, e suas 
crianças estão usando snapchat no lugar dele. Curvas exponenciais tem uma 
tendência de dar um tapa na sua cara antes que você posse vê-las chegar.

O Bitcoin é uma tecnologia exponencial, construída em cima de tecnologias exponenciais.
O site \textit{Our World in Data}\footnote{\url{https://ourworldindata.org/}}
lindamente mostra a velocidade crescente de adoção das tecnologias, começando em 1903
com a introdução de linhas de telefone (veja a figura~\ref{fig:tech-adoption}).
Linhas de telefone, eletricidade, computadores, internet, telefones celulares; 
todos seguem uma tendência exponencial de custo-performance e adoção. 
O Bitcoin também segue essa tendência~\cite{tech-adoption}.

\begin{figure}
  \includegraphics{assets/images/tech-adoption.png}
  \caption{O Bitcoin está literalmente rompendo dos gráficos}
  \label{fig:tech-adoption}
\end{figure}

Bitcoin não só tem um, mas múltiplos efeitos de rede\footnote{Trace Mayer,
  \textit{The Seven Network Effects of Bitcoin}~\cite{7-network-effects}}, todos
resultando crescimentos exponenciais nas suas respectivas áreas: preço, usuários, 
segurança, desenvolvedores, fatia do mercado e adoção global como dinheiro.

Tendo sobrevivido a sua infância, o Bitcoin continua crescendo todo dia em vários aspectos.
Claro, a tecnologia ainda não alcançou a maturidade, ainda. Talvez esteja na sua 
adolescência. Mas se a tecnologia é exponencial, o caminho da obscuridade até 
o uso cotidiano é curto. 

\begin{figure}
  \includegraphics{assets/images/mobile-phone.png}
  \caption{Telefone Celular, ca 1965 vs 2019.}
  \label{fig:mobile-phone}
\end{figure}

Jeff Bezos em sua palestra de 2003 no TED talk, escolheu usar a eletricidade 
como metáfora para o futuro da internet.\footnote{\url{http://bit.ly/bezos-web}} 
Todos os três fenômenos, internet, Bitcoin --- são tecnologias \textit{habilitadoras}, 
redes que habilitam outras coisas. Elas são uma infraestrutura que terá várias coisas construídas utilizando-a como base. É fundadora por natureza.

A eletricidade tem estado um bom tempo entre nós. Nós tomamos ela por garantida.
A internet é bem mais jovem, mas a maioria das pessoas já a considera por 
garantida também. O Bitcoin tem apenas dez anos, mas entrou na consciência 
do público durante a febre do último ciclo. Apenas quem adotou primeiro é 
quem o toma por garantido. Quanto mais tempo passar, mais pessoas vão reconhecer 
o Bitcoin como uma coisa que simplesmente é.\footnote{Isso é chamado de
  \textit{Efeito Lindy}. O Efeito Lindy é uma teoria onde a expectativa 
  de vida de coisas não perecíveis, coisas como tecnologia, ou uma ideia, 
  é proporcional a sua idade atual, onde cada período de sobrevivência adicional 
  implica em uma longevidade maior.~\cite{wiki:lindy}}

Em 1994, a internet ainda era confusa e não intuitiva. Olhando esse gravação antiga
do programa  \textit{Today Show}\footnote{\url{https://youtu.be/UlJku_CSyNg}} nos mostra que o que é 
natural e intuitivo agora, na verdade não era no passado. O Bitcoin ainda é confuso e alienígena 
para muitos, mas assim como a internet agora é uma 'segunda natureza' para os 'nativos digitais', 
gastar e empilhar sats\footnote{\url{https://twitter.com/hashtag/stackingsats}} vai ser algo 
cotidiano para os nativos de bitcoin no futuro.

\begin{quotation}\begin{samepage}
\enquote{O Futuro já chegou --- só não está igualmente bem distribuído.}
\begin{flushright} -- William Gibson\footnote{William Gibson, \textit{The Science in Science Fiction} \cite{william-gibson}}
\end{flushright}\end{samepage}\end{quotation}

Em 1995, quase $15\%$ dos adultos Americanos usaram a internet. 
Dados históricos da Pew Research Center~\cite{pew-research} nos mostram como a internet está 
emaranhada nas nossas vidas. De acordo com uma pesquisa dos consumidores feita pelo 
Kaspersky Lab~\cite{web:kaspersky}, 13\% dos correspondentes usaram Bitcoin e os seus clones para pagar por 
bens em 2018. Enquanto pagamentos não são o único caso de uso do bitcoin, existe alguns indicativos de onde 
estamos, falando como se a internet tivesse um cronograma: estamos no início dos anos 90.

Em 1997, Jeff Bezos disse em uma carta para os acionistas~\cite{bezos-letter} que
\enquote{Este é o Dia 1 para a Internet}, reconhecendo o poder ainda não explorado
do potencial da rede, e, por extensão, da sua empresa. Qualquer que seja o dia hoje para o Bitcoin, 
todo o potencial não explorado é visto por todos, menos aos observadores casuais. 

\begin{figure}
  \includegraphics{assets/images/internet-evolution-white-dates.png}
  \caption{A internet, 1982 vs 2005. Imagem sob licença cc-by-sa da Merit Network, Inc. and Barrett Lyon, Opte Project}
  \label{fig:internet-evolution-white-dates}
\end{figure}

O primeiro node de Bitcoin foi ligado em 2009, após Satoshi ter minerado 
\textit{o bloco gênesis}\footnote{O bloco gênesis é o primeiro bloco da blockchain do Bitcoin.
  Versões mais modernas do cliente Bitcoin a enumeram como bloco $0$, embora em versões bem iniciais
  fosse contada como bloco 1. O Bloco gênesis é costumeiramente codificado nas aplicações de software 
  que utilizam a blockchain do Bitcoin. É um tipo especial de caso que não faz referência a um bloco anterior 
  e produz um subsídio que não pode ser gasto. O parâmetro \textit{coinbase} contêm, além de dados normais, o seguinte texto:
  \textit{\enquote{The Times 03/Jan/2009 Chancellor on brink of second bailout for banks}} \cite{btcwiki:genesis-block}}
e divulgou o código para o público. O seu node não ficou sozinho por muito tempo. Hal Finney foi uma das primeiras pessoas
a entender a ideia e se juntou a rede. Dez anos depois, enquanto eu escrevo, mais de 
$75.000$\footnote{\url{https://bit.ly/luke-nodecount}} nodes estão rodando o bitcoin.

\begin{figure}
  \centering
  \includegraphics[width=8cm]{assets/images/running-bitcoin.png}
  \caption{Hal Finney escreveu o primeiro tweet mencionando o bitcoin em Janeiro 2009.}
  \label{fig:running-bitcoin}
\end{figure}

A camada do protocolo base não é a única coisa que cresce exponencialmente.
A lightning network, uma tecnologia de segunda camada, está crescendo a um passo ainda mais rápido.

Em Janeiro de 2018, a lightning network tinha $40$ nodes e $60$ canais~\cite{web:lightning-nodes}.
Em Abril de 2019, a rede cresceu para mais de $4000$ nodes e por volta de $40.000$ canais.
Isso tudo ainda sendo uma tecnologia experimental onde perda de fundos podem acontecer. 
Ainda assim, a tendência é clara: Milhões de pessoas são ousadas e estão ávidas para usá-la.

\begin{figure}
  \includegraphics{assets/images/lnd-growth-lopp-white.png}
  \caption{Lightning Network, Janeiro de 2018 vs Dezembro de 2018 Fonte: Jameson Lopp}
  \label{fig:lnd-growth-lopp-white.png}
\end{figure}

Para mim, tendo vivido a subida meteórica da internet, os paralelos 
entre ela e o Bitcoin são óbvios. Ambas são redes, ambas são tecnologias 
exponenciais, e ambas permitem novas possibilidades, novas indústrias 
novos modos de vida. Assim como a eletricidade era a melhor metáfora 
para entender onde a internet estava indo, a internet pode ser a melhor 
metáfora para entender onde o bitcoin está indo. Ou, nas palavras de 
Andreas Antonopoulos, o Bitcoin é \textit{A Internet do Dinheiro}.
Essas metáforas ajudam a nos lembrar que a história não se repete, mas muitas vezes rima.

Tecnologias experimentais são difíceis de entender e muitas vezes subestimadas.
Mesmo que eu tenha um grande interesse por essas tecnologias, eu sou constantemente 
surpreendido pela rapidez do progresso e da inovação. Ver o ecossistema do Bitcoin crescer, 
é como ver o crescimento da internet de maneira acelerada. É extremamente estimulante.

Minha busca por tentar entender o Bitcoin me levou para muitos caminhos na história, e às vezes, para mais de um.
Entender sociedades antigas, dinheiro do passado, e como redes de comunicações se desenvolveram, 
foi tudo parte da jornada. Do machado para o smartphone, a tecnologia realmente mudou nosso mundo muitas vezes.
Tecnologias de redes são especialmente transformadoras, a escrita, as estradas, a eletricidade, a internet. 
Todas elas mudaram o mundo. O Bitcoin mudou o meu mundo e vai continuar mudando a mente e os corações 
de todos os que ousarem usá-lo.

\paragraph{O Bitcoin me ensinou que entender o passado é essencial para conhecer o futuro. Um futuro que apenas começou\ldots}

% ---
%
% #### Down the Rabbit Hole
%
% - [The Rising Speed of Technological Adoption][the rising speed of technological adoption] by Jeff Desjardins
% - [The 7 Network Effects of Bitcoin][multiple network effects] by Trace Mayer
% - [The Electricity Metaphor for the Web's Future][TED talk] by Jeff Bezos
% - [How the internet has woven itself into American life][data from the Pew Research Center] by Susannah Fox and Lee Rainie
% - [Genesis Block][genesis block] on the Bitcoin Wiki
% - [Lindy Effect][more time] on Wikipedia
%
% [Our World in Data]: https://ourworldindata.org/
% [the rising speed of technological adoption]: https://www.visualcapitalist.com/rising-speed-technological-adoption/
% [multiple network effects]: https://www.thrivenotes.com/the-7-network-effects-of-bitcoin/
% [TED talk]: https://www.ted.com/talks/jeff_bezos_on_the_next_web_innovation
% [recording of the Today Show]: https://www.youtube.com/watch?v=UlJku_CSyNg
% [William Gibson]: https://www.npr.org/2018/10/22/1067220/the-science-in-science-fiction
% [data from the Pew Research Center]: https://www.pewinternet.org/2014/02/27/part-1-how-the-internet-has-woven-itself-into-american-life/
% [consumer survey]: https://www.kaspersky.com/blog/money-report-2018/
% [letter to shareholders]: http://media.corporate-ir.net/media_files/irol/97/97664/reports/Shareholderletter97.pdf
% [running bitcoin]: https://twitter.com/halfin/status/1110302988?lang=en
% [40 nodes]: https://bitcoinist.com/bitcoin-lightning-network-mainnet-nodes/
% [reckless]: https://twitter.com/hashtag/reckless
% [Jameson Lopp]: https://twitter.com/lopp/status/1077200836072296449
% [\textit{The Internet of Money}]: https://theinternetofmoney.info/
% [stacking]: https://twitter.com/hashtag/stackingsats
%
% <!-- Bitcoin Wiki -->
% [genesis block]: https://en.bitcoin.it/wiki/Genesis_block
%
% <!-- Wikipedia -->
% [more time]: https://en.wikipedia.org/wiki/Lindy_effect
% [alice]: https://en.wikipedia.org/wiki/Alice%27s_Adventures_in_Wonderland
% [carroll]: https://en.wikipedia.org/wiki/Lewis_Carroll

\addpart{Pensamentos Finais}

\pdfbookmark{Conclusão}{Conclusão}
\label{ch:conclusion}

\chapter*{Conclusão}

\begin{chapquote}{Lewis Carroll, \textit{Alice in Wonderland}}
\enquote{Comece pelo começo}, disse o Rei, gravemente, \enquote{e prossiga até chegar ao fim; então pare}.

\end{chapquote}
Como mencionado no início, eu acho que qualquer resposta para a questão 
\textit{“O que você aprendeu com o Bitcoin?”} vai ser sempre incompleta. 
A simbiose de vários sistemas que podem ser vistos como vivos -- Bitcoin, 
a tecnoesfera, e economia -- é muito interligada, os tópicos muito numerosos, e 
as coisas estão se movendo muito rápido para uma pessoa sozinha entender tudo.

Como mencionei no início, acho que qualquer resposta à questão \textit{“O que você aprendeu com o Bitcoin?”} sempre estará incompleta. A simbiose do que pode ser visto como múltiplos sistemas vivos - Bitcoin, a tecnosfera e a economia - é muito interligada, os tópicos muito numerosos e as coisas estão se movendo muito rápido para serem totalmente compreendidas por uma única pessoa.

Mesmo sem entendê-lo totalmente, e mesmo com todas as suas peculiaridades e aparentes deficiências, o Bitcoin sem dúvida, funciona. Ele continua produzindo blocos aproximadamente a cada dez minutos e faz isso lindamente. Quanto mais tempo o Bitcoin continua funcionando, mais pessoas optam por usá-lo.

\begin{quotation}\begin{samepage}
\enquote{É verdade que as coisas são bonitas quando funcionam. Arte é funcionamento.}
\begin{flushright} -- Giannina Braschi\footnote{Giannina Braschi, \textit{O Empério dos Sonhos} \cite{braschi2011empire}}
\end{flushright}\end{samepage}\end{quotation}

\paragraph{}
O Bitcoin é um filho da Internet. Está crescendo exponencialmente, confundindo as linhas entre as disciplinas. Não está claro, por exemplo, onde termina o reino da tecnologia pura e onde começa outra matéria. Embora o Bitcoin exija que os computadores funcionem com eficiência, a ciência da computação não é suficiente para entendê-lo. O Bitcoin não é apenas sem fronteiras no que diz respeito ao seu funcionamento interno, mas também sem fronteiras no que diz respeito às disciplinas acadêmicas.

Economia, política, teoria dos jogos, história monetária, teoria das redes, finanças, criptografia, teoria da informação, censura, lei e regulamentação, organização humana, psicologia --- tudo isso e mais, são áreas de especialização que podem ajudar na busca pelo conhecimento de como o Bitcoin funciona e o que é Bitcoin.

Nenhuma invenção é responsável por seu sucesso. É a combinação de várias peças, que antes não se relacionavam, unidas por incentivos da teoria dos jogos, que constituem a revolução que é o Bitcoin. A bela mistura de muitas disciplinas é o que torna Satoshi um gênio.

\paragraph{}
Como todo sistema complexo, o Bitcoin precisa fazer concessões em termos de eficiência, custo, segurança e muitas outras propriedades. Assim como não existe uma solução perfeita para derivar um quadrado de um círculo, qualquer solução para os problemas que o Bitcoin tenta resolver sempre será imperfeita também.

\begin{quotation}\begin{samepage}
\enquote{Não acredito que algum dia teremos um bom dinheiro novamente antes de tirarmos isso das mãos do governo, ou seja, não podemos tirá-lo violentamente das mãos do Estado, tudo o que podemos fazer é por alguma astuta forma indireta, introduzir algo que eles não podem parar.}
\begin{flushright} -- Friedrich Hayek\footnote{Friedrich Hayek sobre Política Monetária, o Padrão Ouro, Déficits, Inflação, e John Maynard Keynes \url{https://youtu.be/EYhEDxFwFRU}}
\end{flushright}\end{samepage}\end{quotation}

O bitcoin é a maneira astuta e indireta de reintroduzir um bom dinheiro ao mundo. Ele faz isso colocando um indivíduo soberano atrás de cada node, assim como Da Vinci tentou resolver o problema intratável de enquadrar em um círculo, colocando o Homem Vitruviano em seu centro. Os nós removemos com eficácia qualquer conceito de centro, criando um sistema surpreendentemente antifrágil e extremamente difícil de ser desligado. O Bitcoin vive, e seu batimento cardíaco provavelmente durará mais que o nosso.

Espero que você tenha gostado dessas vinte e uma lições. Talvez a lição mais importante seja que o Bitcoin deve ser examinado holisticamente, de vários ângulos, se alguém quiser ter algo próximo de uma imagem completa. Assim como remover uma parte de um sistema complexo destrói o todo, examinar partes do Bitcoin isoladamente parece contaminar a compreensão dele. Se apenas uma pessoa eliminar a \enquote{blockchain} de seu vocabulário e substituí-la por \enquote{uma cadeia de blocos}, morrerei feliz.

Em qualquer caso, minha jornada continua. Pretendo me aventurar ainda mais nas profundezas desta toca do coelho e convido você a acompanhá-lo no passeio.\footnote{\url{https://twitter.com/dergigi}}

% <!-- Twitter -->
% [dergigi]: https://twitter.com/dergigi
%
% <!-- Internal -->
% [sly roundabout way]: https://youtu.be/EYhEDxFwFRU?t=1124
% [Giannina Braschi]: https://en.wikipedia.org/wiki/Braschi%27s_Empire_of_Dreams


\cleardoublepage

\chapter*{Agradecimentos}
\pdfbookmark{Acknowledgments}{acknowledgments}

Agradeço aos incontáveis autores e produtores de conteúdo que influenciaram meu pensamento sobre o Bitcoin e os tópicos que ele aborda. Há muitos de vocês para listar, mas farei o meu melhor para citar alguns.

\begin{itemize}
  \item Obrigado a Arjun Balaji pelo tweet que me motivou a escrever este livro.
  \item Obrigado a Marty Bent por fornecer conteúdo sem fim para reflexão e entretenimento. Se você não segue Marty’s Bent e o Tales From The Crypt, não sabe o que está perdendo. Felicidades Matt e Marty por nos guiarem pela toca do coelho
  \item Obrigado a Michael Goldstein e Pierre Rochard pela curadoria e fornecimento da melhor literatura Bitcoin por meio do Instituto Nakamoto. E obrigado por criar o Noded Podcast, que influenciou substancialmente minhas visões filosóficas sobre o Bitcoin.
  \item Obrigado a Saifedean Ammous por suas convicções, tweets picantes e por escrever o padrão Bitcoin.
  \item Obrigado a Francis Pouliot por compartilhar sua empolgação em descobrir sobre a cadeia do tempo.
  \item Obrigado a Andreas M. Antonopoulos por todo o material didático que vem publicando ao longo dos anos.
  \item Obrigado a Peter McCormack por seus tweets honestos e pelo podcast What Bitcoin Did, que continua fornecendo ótimas percepções de muitas áreas do saber.
  \item Obrigado a Jannik, Brandon, Matt, Camilo, Daniel, Michael e Raphael por fornecerem feedback para os primeiros rascunhos de algumas lições. Agradecimentos especiais a Jannik que revisou vários rascunhos várias vezes.
  \item Obrigado a Dhruv Bansal e Matt Odell por dedicar seu tempo para discutir algumas dessas ideias comigo.
  \item Obrigado a Guy Swann por produzir uma versão em áudio de 21lessons.com.
  \item Obrigado ao Friar Hass por seu apoio espiritual e orientação, e por dedicar seu tempo para escrever um prefácio para este livro.
  \item Obrigado a minha esposa por me tolerar minha natureza obsessiva.
  \item Obrigado à minha família por me apoiar tanto nos momentos bons como nos maus.
  \item Por último, mas não menos importante, graças a todos os bitcoinheiros maximalistas, shitcoinheiros minimalistas, shills, bots e shitposters que residem no belo jardim que é o Bitcoin Twitter.
\end{itemize}

E, finalmente, obrigado por ler isso. Espero que você tenha gostado tanto quanto eu gostei de escrever.

\listoffigures

\chapter*{Sobre a Bibliografia}
\pdfbookmark{Bibliography}{bibliography}

Hoje, muitos livros foram publicados sobre o Bitcoin. No entanto, a maioria das conversa --- e, portanto, a maioria dos recursos mais interessantes --- acontecem online.

\paragraph{}
A bibliografia a seguir lista livros, artigos e recursos online. Se o recurso tem um URL associado a ela, a URL estava ativa e funcionando em Outubro de 2019, pois consegui acessar com sucesso o dito recurso. Se qualquer uma das URLs levar a uma página inativa, sinto muito. Por favor deixe-me saber disso pelo \footnote{\url{https://dergigi.com/contact}} para que eu possa atualizar o(s) link(s).

\paragraph{}
P.S: O Bitcoin e o \href{https://ipfs.io/}{IPFS} resolvem isso.

\bibliography{main}

\end{document}
